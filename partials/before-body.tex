%%%%% CHOOSE YOUR LINE SPACING HERE
% This is approx. 1.5 spacing:
\setlength{\textbaselineskip}{18pt plus2pt minus1pt}

% You can set the spacing here for the roman-numbered pages (acknowledgements, table of contents, etc.)
\setlength{\frontmatterbaselineskip}{18pt plus2pt minus1pt}

% Leave this line alone; it gets things started for the real document.
\setlength{\baselineskip}{\textbaselineskip}

%%%%% CHOOSE YOUR SECTION NUMBERING DEPTH HERE
% You have two choices.  First, how far down are sections numbered?  (Below that, they're named but
% don't get numbers.)  Second, what level of section appears in the table of contents?  These don't have
% to match: you can have numbered sections that don't show up in the ToC, or unnumbered sections that
% do.  Throughout, 0 = chapter; 1 = section; 2 = subsection; 3 = subsubsection, 4 = paragraph...

% The level that gets a number:
\setcounter{secnumdepth}{$secnumdepth$}
% The level that shows up in the ToC:
\setcounter{tocdepth}{$tocdepth$}

%%%%% begin titlepage extension code

$if(has-frontmatter)$
  \begin{frontmatter}
$endif$

\begin{titlepage}
$-- % Coverpage
$if(coverpage-true)$
$_coverpage.tex()$
$endif$
$if(coverpage-include-file)$

$for(coverpage-include-file)$\input{$coverpage-include-file$}
\clearpage
$endfor$$endif$

$-- % Titlepage
$if(titlepage-true)$
$if(titlepage-file)$
% Use the file
\input{$titlepage-filename$}
$else$
$_titlepage.tex()$
$endif$
$endif$
$if(titlepage-include-file)$

$for(titlepage-include-file)$\input{$titlepage-include-file$}
\clearpage
$endfor$$endif$
\end{titlepage}
% \setcounter{page}{1}
$if(has-frontmatter)$
\end{frontmatter}
$endif$

%%%%% end titlepage extension code


%%%%% ABSTRACT SEPARATE
% This is used to create the separate, one-page abstract that you are required to hand into the Exam
% Schools.  You can comment it out to generate a PDF for printing or whatnot.
% \begin{abstractseparate}
% 	\input{frontmatter/03-abstract.txt} % Create an abstract.tex file in the 'text' folder for your abstract.
% \end{abstractseparate}

% \newpage
% \thispagestyle{empty}
% \null
% \newpage
% Pages are roman numbered from here, though page numbers are invisible until ToC.  This is in
% keeping with most typesetting conventions.
\pagenumbering{roman}

\begin{romanpages}

% Title page is created here
\maketitle

%%%%% SUMMARY

\begin{summary}
\addcontentsline{toc}{chapter}{Summary}
	\input{frontmatter/summary.qmd}
\end{summary}

%%%%% SUMMARY TRANSLATION

\begin{summarytranslation}
%\addcontentsline{toc}{chapter}{Zusammenfassung}
	\input{frontmatter/summary_translation.qmd}
\end{summarytranslation}

%%%%% ACKNOWLEDGEMENTS 

\begin{acknowledgements}
\addcontentsline{toc}{chapter}{Acknowledgements}
 	\input{frontmatter/acknowledgements.qmd}
\end{acknowledgements}

% This aligns the bottom of the text of each page.  It generally makes things look better.
\flushbottom

